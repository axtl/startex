%!TEX root = main.tex

% This glossary uses the handy \latexpackage{acroynym} package to automatically
% maintain the glossary.  It uses the package's \texttt{printonlyused}
% option to include only those acronyms explicitly referenced in the
% \LaTeX\ source.

% use \acrodef to define an acronym, but no listing
\acrodef{AES}{Advanced Encryption Standard}
\acrodef{SPRNG}{Secure Pseudo-Random Number Generator}

% \begin{acronym}
    % \acro{SPRNG}{Secure Pseudo-Random Number Generator}
    % \acro{RSA}{Rivest, Shamir, \& Adleman\acroextra{ encryption}}
    % \acro{RSAES-OAEP}{RSA Encryption Scheme\acroextra{ with \acl{OAEP}}}
% \end{acronym}


% The acronym environment will typeset only those acronyms that were
% *actually used* in the course of the document

% You can also use \newacro{}{} to only define acronyms
% but without explictly creating a glossary
%
% \newacro{ANOVA}[ANOVA]{Analysis of Variance\acroextra{, a set of
%   statistical techniques to identify sources of variability between groups.}}
% \newacro{API}[API]{application programming interface}
% \newacro{GOMS}[GOMS]{Goals, Operators, Methods, and Selection\acroextra{,
%   a framework for usability analysis.}}
% \newacro{TLX}[TLX]{Task Load Index\acroextra{, an instrument for gauging
%   the subjective mental workload experienced by a human in performing
%   a task.}}
% \newacro{UI}[UI]{user interface}
% \newacro{UML}[UML]{Unified Modelling Language}
% \newacro{W3C}[W3C]{World Wide Web Consortium}
% \newacro{XML}[XML]{Extensible Markup Language}
