% paper.tex template - based off template.la from
% http://www.usenix.org/events/osdi08/cfp/requirements.html

% [final] directive brings in any pictures you have
% remove or replace with [draft] to suppress that before you're complete if you prefer
\documentclass[letterpaper,twocolumn,10pt,final]{article}
\usepackage{style}
\begin{document}

%don't want date printed
\date{}

%make title bold and 14 pt font (Latex default is non-bold, 16 pt)
\title{\fontfamily{phv}\selectfont
    {\huge{\textbf{StarTeX}}}\\
    {\large{\textbf{\\A LaTeX Starter}}}}


%for single author (just remove % characters)
\author{
{\rm \textbf{First Last}}\\
{\rm email@provider.tld}\\
Organization\\
% \and
% {\rm \textbf{That Other Person}\\
% {\rm email@example.edu}\\
%Name Institution
} % end author

\maketitle

% Use the following at camera-ready time to suppress page numbers.
% Comment it out when you first submit the paper for review.
\thispagestyle{empty}

\begin{abstract}
This is \textbf{startex}~\cite{startex}~\endnote{The tilde character (\~{}) in the source means a non-breaking space. This way, your reference will always be attached to the word that preceded it, instead of going to the next line. Useful for citations and endnotes as well.}, a LaTeX starter for conference-style papers, based off the USENIX template. It includes some of the default styles one might use, and some hints on how to use them (not that I'm an expert).

Comments on how various things work are in the source. Don't hesitate to send a pull request if you know of better ways to do some of the things herein. \textit{Thanks!}
\end{abstract}

% It's always a good idea to split the document up in files, maybe one per section; I don't do that here for your ease of learning, but you should.


% use \label{sec:secname} for consistency and to make it easy to \ref{sec:secname} where you need to. The {hyperref} package makes nice links in PDFs.
\section{\label{sec:intro}Introduction}
This paper is organized as follows: Section~\ref{sec:intro} is what you're reading. Section~\ref{sec:code} shows some code. Section~\ref{sec:ft} shows figures and tables. Section~\ref{sec:tex} tells you about compiling LaTeX into PDFs. That's about it.


\section{\label{sec:code}Code}
Some embedded literal typeset code might look like the following:
\\ % forced linebreaks
\texttt{
    \small\lstinputlisting[language=Python]{code.py}
}


\section{\label{sec:ft}Figures and Tables}
Getting pictures and tables exactly where you want them is a bit of trial-and-error, so it's particularly useful to have them in separate files so you can easily move them around. Go forth and experiment\ldots

% !TEX root = main.tex
\begin{figure}[h!]
    \centering
        % actual name of file minus extension inside {}
        \includegraphics[width=0.475\textwidth]{fig}
    % caption - what's the fig about? move above \include if you prefer
    \caption{A figure}
    % labels make it nice to refer back to figures
    \label{fig:figure1}
\end{figure}

% !TEX root = main.tex
\begin{table*}
    \centering
    % columns; can use multipliers as well
    \begin{tabular}{l*{8}c}
        \toprule
        % multicolumn allows cell merging for headings; specify which cells to create the multi-column for
        \multicolumn{1}{c}{\textbf{Server}} & \multicolumn{6}{c}{\textbf{GET}} & \multicolumn{2}{c}{\textbf{POST}} \\
        \cmidrule{2-9} % specify which columns to draw line for
        & \multicolumn{2}{c}{100-all} & \multicolumn{2}{c}{100-single} & \multicolumn{2}{c}{1000-single} & \multicolumn{2}{c}{1000} \\
        & S & L & S & L & S & L & S & L \\
        \midrule
        \emph{SRV\_A} & 4.83 & 4.96 & 838.08 & 413.71 & 720.86 & 673.97 & 654.51 & 763.50 \\
        \emph{SRV\_B} & 2.85 & 2.96 & 99.45 & 113.24 & 142.75 & 137.62 & 142.88 & 123.82 \\
        \bottomrule
    \end{tabular}
    \caption{A table (of requests/second)}
    % your legend can be another table, too
    \caption*{\textbf{GET} Legend}
        \begin{tabular}{*{2}l}
            \toprule
            \emph{100-all} & retrieve \texttt{/posts/} 100 times \\
            \emph{100-single} & retrieve \texttt{/posts/123/} 100 times \\
            \emph{1000-single} & retrieve \texttt{/posts/234/} 1000 times \\
            \bottomrule
        \end{tabular}
    \label{tab:eval}
\end{table*}



\section{\label{sec:tex}Getting the PDF}
There are a number of ways to compile LaTeX, but I \textbf{strongly} recommend using the \texttt{Makefile}~\cite{latex-makefile} included in this package.

If you have multiple files included by a \textit{main} one, (which you really should) you specify the name of the top-level \texttt{.tex} file when you run \texttt{make}, and the rest will be taken care of for you. So in this repo, you'd run \texttt{make paper} to build the PDF.

{\footnotesize 
    \bibliographystyle{acm}
    \bibliography{bibliography}
    % \bibliography{otherbib} can have multiple bib files, of course
}

% comment out if you don't have endnotes anywhere
\theendnotes

\end{document}
