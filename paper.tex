% paper.tex template - based off template.la from
% http://www.usenix.org/events/osdi08/cfp/requirements.html

% [final] directive brings in any pictures you have
% remove or replace with [draft] to suppress that before you're complete if you prefer
\documentclass[letterpaper,twocolumn,10pt,final]{article}
\usepackage{style}
\begin{document}

%don't want date printed
\date{}

%make title bold and 14 pt font (Latex default is non-bold, 16 pt)
\title{\fontfamily{phv}\selectfont
    {\huge{\textbf{StarTeX}}}\\
    {\large{\textbf{\\A LaTeX Starter}}}}


%for single author (just remove % characters)
\author{
{\rm \textbf{First Last}}\\
{\rm email@provider.tld}\\
Organization\\
% \and
% {\rm \textbf{That Other Guy}\\
% {\rm email@example.edu}\\
%Name Institution
} % end author

\maketitle

% Use the following at camera-ready time to suppress page numbers.
% Comment it out when you first submit the paper for review.
\thispagestyle{empty}

\begin{abstract}
This is \textbf{startex}~\cite{startex}, a LaTeX starter for conference-style papers, based off the USENIX template. It includes some of the default styles one might use, and some hints on how to use them (not that I'm an expert).

Please refer to the source in \texttt{paper.tex} (if you aren't already) for comments on what the different parts do. Don't hesitate to send updates if you know of better ways to do some of the things herein. \textit{Thanks!}
\end{abstract}

\section{\label{sec:intro}Introduction}

% use \label{sec:secname} for consistency and to make it easy to \ref{sec:secname} where you need to. The {hyperref} package makes nice links in PDFs.

This paper is organised as follows: Section~\ref{sec:intro} is what you're reading. Section~\ref{sec:figs} shows a figure. Section~\ref{sec:code} shows some code. That's about it.
\\
\\
You can also have endnotes\endnote{But if you don't, you \textbf{MUST} comment out the \texttt{\\theendnotes} line at the end of the document.}. 
\\
\\
And of course, you must cite lots to have a good paper. Often, peer reviewers will give a baseline review, then multiply it by the number of references you have. You generally want to find a way to cite \textit{Knuth}, \textit{Ritchie} or \textit{Diffie}~\cite{Diffie1976} for extra bonus.
\\
\\
The tilde character (\~{}) in the source means a non-breaking space. This way, your reference will always be attached to the word that preceded it, instead of going to the next line. Useful for references as well.

\section{\label{sec:figs}Figures}

You need to look at the code for this. Notice the picture doesn't go exactly where you want it---but you can do your hardest to force it!

\begin{figure}[h!]
    \centering
        % actual name of file minus extension inside {}
        \includegraphics[width=0.475\textwidth]{fig}
    % caption - what's the fig about? move above \include if you prefer
    \caption{A figure}
    % labels make it nice to refer back to figures
    \label{fig:figure1}
\end{figure}

A trick I sometimes use is to define figs in a separate file (say, \texttt{figures.sty}) and include it in the \texttt{usepackage} directive at the top of the document. Helpful when you have lots of pictures and want to make moving them around easier.

\section{\label{sec:code}Some Code}

\subsection{\label{sec:code:py}Code Listings}

Some embedded literal typeset code might look like the following :

\texttt{
    \small\lstinputlisting[language=Python]{code.py}
}

The above uses the \texttt{\{listings\}} and \texttt{\{color\}} packages. Don't worry, it's all taken care for you in \texttt{style.sty}.

\subsection{\label{sec:code:tex}LaTeX Magic}

Did you look at the code in \texttt{paper.tex} yet? Hopefully, you're actually reading that.

There are complicated ways of compiling LaTeX stuff, but I recommend using the awesome \texttt{Makefile}~\cite{latex-makefile} (included in the repo). It'll spit out a PDF on the other end and take care of everything and the kitchen sink for you.

\subsubsection*{Suppressing Section Numbers}

% \ldots for ellipsis, don't use three dots - poor spacing.
You do so using \texttt{(sub\ldots)section*}

{\footnotesize 
    \bibliographystyle{acm}
    \bibliography{bibliography}}

% comment out if you don't have endnotes anywhere
\theendnotes

\end{document}







